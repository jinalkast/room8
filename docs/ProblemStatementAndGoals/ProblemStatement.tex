\documentclass[12pt, titlepage]{article}

\usepackage{booktabs}
\usepackage{float}
\usepackage{tabularx}
\usepackage{hyperref}
\hypersetup{
    colorlinks,
    citecolor=black,
    filecolor=black,
    linkcolor=red,
    urlcolor=blue
}
\usepackage[round]{natbib}
\usepackage{longtable}
\usepackage[shortlabels]{enumitem}
\usepackage{hyperref}
\hypersetup{
    colorlinks,
    citecolor=blue,
    filecolor=black,
    linkcolor=red,
    urlcolor=blue
}
\usepackage{amsmath}
\usepackage{array}
\usepackage{graphicx}
\usepackage[paper=a4paper]{geometry}

\input{../Comments}
%% Common Parts

\newcommand{\progname}{Room8} % PUT YOUR PROGRAM NAME HERE
\newcommand{\authname}{Team 19
\\ Mohammed Abed
\\ Maged Armanios
\\ Jinal Kasturiarachchi
\\ Jane Klavir
\\ Harshil Patel} % AUTHOR NAMES

\newcommand{\testcase}[1]{
    \begin{tabular}{|p{3cm} p{10cm}|}
    \hline
    #1
    \hline
    \end{tabular}
    \vspace{0.5cm} % Adds a 0.5 cm gap between tables
}

\usepackage{hyperref}
    \hypersetup{colorlinks=true, linkcolor=blue, citecolor=blue, filecolor=blue,
                urlcolor=blue, unicode=false}
    \urlstyle{same}
                                




\begin{document}

\title{Problem Statement and Goals: Room8} 
\author{\authname}
\date{\today}
	
\maketitle

\pagenumbering{roman}


\section{Revision History}

\begin{tabularx}{\textwidth}{p{3cm}p{2cm}X}
\toprule {\bf Date} & {\bf Version} & {\bf Notes}\\
\midrule
2024-09-24 & Team & Version 0.0 \\
2025-01-08 & Maged & Implementing TA feedback and changed extras - Version 0.1\\
2025-04-10 & Harshil & Revising doc to reflect TA feedback - Version 1.0\\
\bottomrule
\end{tabularx}

~\newpage


\section{Problem Statement}
The overview of problems Room8 will solve and information on the inputs, outputs, stakeholders and the system environment.
\subsection{Problem}
A common experience that the vast majority of the human population encounters is a shared living arrangement, whether that be with family members or, prevalent in post-secondary studies, roommates. This scenario offers a number of challenges, including the maintenance of shared spaces such as a kitchen or living room. As the power dynamic between roommates are typically equal, it can often feel cumbersome to reach out to a roommate and complain about the mess that they leave behind in a common space. This friction can be avoided by setting a standard for the cleanliness that is expected of any and all members of the same residence. Room8 aims to deliver an application, which assesses the cleanliness of a shared space. Additionally, the application will include features that will benefit users who have roommates by addressing other issues that may rise from the living arrangement, such as shared expenses, scheduling activities and chores, and encouraging communication between members.

\subsection{Inputs and Outputs}
Relevant inputs and outputs of the software used in the application.
\subsubsection{Inputs}
\begin{itemize}
\item Images of the monitored area.
\item Video feed of monitored area.
\item Numerical data paired with splitting method.
\item True or false field for payments.
\item JSON of text with date.
\item JSON of objects containing object name, person and status.
\end{itemize}
\subsubsection{Outputs}
\begin{itemize}
\item Two images, one before and an after with bounding boxes around changes.
\item JSON of changes made to shared space.
\item Numerical data of amount user owes others and how much the user is owed.
\item JSON calendar data with scheduled events at specified dates.
\item JSON containing history of tasks assigned and completed.
\item JSON containing history of expenses.
\end{itemize}
\subsection{Stakeholders}
The stakeholders of this project are members of any shared living situation, but it is primarily directed towards students who live in student housing/campus residence. Room8’s stakeholders can include students—both off-campus and on-campus. Off-campus students use the app to manage shared living spaces, track cleanliness, and split expenses, while on-campus students benefit from chore scheduling and event coordination for common areas. Other stakeholders include landlords, who want to ensure their properties are well-maintained; resident assistants, who use the scheduling feature to organize events; and university housing committees, which support student living conditions and conflict resolution.

\subsection{Environment}
Project enviornment of both hardware and software components.

\subsubsection{Hardware Environment}
The hardware environment mainly consists of the camera system that will be
installed in the shared space. The camera will be used to capture images of the area, as well as detect when someone enters and/or exists.
\subsubsection{Software Environment}
The software environment is twofold. For the cleanliness detection system, an algorithm will need to be produced in order to assess the space as well as detect motion from the camera feed and take pictures according to camera feedback. Besides that, there must also be a full-stack web and/or mobile application developed, for the assessment to be displayed, along with the other additional functionality that was previously mentioned (bill splitter, scheduler, group chat, etc.)


\section{Goals}
All goals listed below stem from the problem statement and aim to achieve the mission of reducing tension and friction common in shared living arrangements:
\begin{itemize}
\item The system must detect and log changes in cleanliness (e.g., clutter, spills) in shared spaces using motion-triggered image analysis.
\item The app must let roommates assign chores, set deadlines, and mark tasks as complete.
\item The app must track shared bills, split costs automatically, and log payments.
\item The system must notify roommates of overdue chores, unpaid bills, or recurring messes via push notifications. 
\item User data (e.g., payment logs) must be visible only to approved roommates.
\item System will keep history of all tasks and chores completed available for users to see.
\end{itemize}

\section{Stretch Goals}
These goals are extras of the project and are considered the future steps of Room8 after revision one is completed whereas the goals above are the main concern.
\begin{itemize}
\item System should provide item-level results (i.e Items a person leaves/uses in the area) in the output of the algorithm which detects the changes to the cleanliness of a living space.
\begin{itemize}
\item The specificity of the output would make the algorithm as objective as possible, and would lay-out the precise steps users can take to improve their cleanliness.
\end{itemize}
\item The system should allow users to select a cleanliness detection model that adapts to the specific living space and learns its features.
\begin{itemize}
\item Beyond the generic algorithm, having this mechanism would make results more individualized and better-suited for specific environments.
\end{itemize}
\end{itemize}


\section{Challenge Level and Extras}

Challenge Level: General
Extras:
\begin{itemize}
\item User Documentation
\item Usability Testing
\end{itemize}

\newpage{}

\section*{Appendix --- Reflection}
\subsection*{Questions}
\begin{enumerate}
    \item What went well while writing this deliverable? 
    \item What pain points did you experience during this deliverable, and how
    did you resolve them?
    \item How did you and your team adjust the scope of your goals to ensure
    they are suitable for a Capstone project (not overly ambitious but also of
    appropriate complexity for a senior design project)?
\end{enumerate}  
\subsection*{Answers}
\begin{enumerate}
\item Generally speaking, the writing of this deliverable went pretty smoothly. I will speak about this more in the following question, but after getting over the hurdles of defining our project and what we wanted it to be (after much back and forth), it was straightforward to take a look at the different sections of the deliverable and what they were asking, and produce a document that we’re all proud of.
\item A large majority of the pain points that we experienced during this deliverable were due to the
uncertainties that we had coming into this phase of the project. Initially, we were debating between two options for our project, one from the list of potential projects, and the other being this one, our original idea. There was a lot of back and forth, and it took us quite a bit of time to meet with the potential supervisors as well. Eventually, we reached a settling point on this concept, while securing a supervisor for the project as well. This reassurance is what allowed us to focus on defining the scope of the problem, and cleared up the confusion about what content we wanted to deliver in this paper.
\item Initially we had very large ambitions for the project that we wanted to deliver, however, we were able to adjust our expectations after having an initial discussion with our supervisor. Since he has much more domain knowledge than us, we were asked many questions that we were unsure about or still had to explore. This left us asking ourselves even more questions, which helped narrow down the scope of what we want to achieve in our project. To be specific, a main point that we hadn’t discussed at all was the issue of privacy, which seems quite obvious when discussing a camera system that’s installed in an area such as a kitchen, but we have since made many considerations regarding privacy and will continue doing so moving forward.
\end{enumerate}
\end{document}