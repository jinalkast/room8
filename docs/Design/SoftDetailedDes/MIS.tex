\documentclass[12pt, titlepage]{article}

\usepackage{amsmath, mathtools}

\usepackage[round]{natbib}
\usepackage{amsfonts}
\usepackage{amssymb}
\usepackage{graphicx}
\usepackage{colortbl}
\usepackage{xr}
\usepackage{hyperref}
\usepackage{longtable}
\usepackage{xfrac}
\usepackage{tabularx}
\usepackage{float}
\usepackage{siunitx}
\usepackage{booktabs}
\usepackage{multirow}
\usepackage[section]{placeins}
\usepackage{caption}
\usepackage{fullpage}

\hypersetup{
bookmarks=true,     % show bookmarks bar?
colorlinks=true,       % false: boxed links; true: colored links
linkcolor=red,          % color of internal links (change box color with linkbordercolor)
citecolor=blue,      % color of links to bibliography
filecolor=magenta,  % color of file links
urlcolor=cyan          % color of external links
}

\usepackage{array}

\externaldocument{../../SRS/SRS}

\input{../../Comments}
%% Common Parts

\newcommand{\progname}{Room8} % PUT YOUR PROGRAM NAME HERE
\newcommand{\authname}{Team 19
\\ Mohammed Abed
\\ Maged Armanios
\\ Jinal Kasturiarachchi
\\ Jane Klavir
\\ Harshil Patel} % AUTHOR NAMES

\usepackage{hyperref}
    \hypersetup{colorlinks=true, linkcolor=blue, citecolor=blue, filecolor=blue,
                urlcolor=blue, unicode=false}
    \urlstyle{same}
                                


\begin{document}

\title{Module Interface Specification for \progname{}}

\author{\authname}

\date{\today}

\maketitle

\pagenumbering{roman}

\section{Revision History}

\begin{tabularx}{\textwidth}{p{3cm}p{2cm}X}
\toprule {\bf Date} & {\bf Version} & {\bf Notes}\\
\midrule
Date 1 & 1.0 & Notes\\
Date 2 & 1.1 & Notes\\
\bottomrule
\end{tabularx}

~\newpage

\section{Symbols, Abbreviations and Acronyms}

See SRS Documentation at \wss{give url}

\wss{Also add any additional symbols, abbreviations or acronyms}

\newpage

\tableofcontents

\newpage

\pagenumbering{arabic}

\section{Introduction}

The following document details the Module Interface Specifications for
\wss{Fill in your project name and description}

Complementary documents include the System Requirement Specifications
and Module Guide.  The full documentation and implementation can be
found at \url{...}.  \wss{provide the url for your repo}

\section{Notation}

\wss{You should describe your notation.  You can use what is below as
  a starting point.}

The structure of the MIS for modules comes from \citet{HoffmanAndStrooper1995},
with the addition that template modules have been adapted from
\cite{GhezziEtAl2003}.  The mathematical notation comes from Chapter 3 of
\citet{HoffmanAndStrooper1995}.  For instance, the symbol := is used for a
multiple assignment statement and conditional rules follow the form $(c_1
\Rightarrow r_1 | c_2 \Rightarrow r_2 | ... | c_n \Rightarrow r_n )$.

The following table summarizes the primitive data types used by \progname. 

\begin{center}
\renewcommand{\arraystretch}{1.2}
\noindent 
\begin{tabular}{l l p{7.5cm}} 
\toprule 
\textbf{Data Type} & \textbf{Notation} & \textbf{Description}\\ 
\midrule
character & char & a single symbol or digit\\
integer & $\mathbb{Z}$ & a number without a fractional component in (-$\infty$, $\infty$) \\
natural number & $\mathbb{N}$ & a number without a fractional component in [1, $\infty$) \\
real & $\mathbb{R}$ & any number in (-$\infty$, $\infty$)\\
\bottomrule
\end{tabular} 
\end{center}

\noindent
The specification of \progname \ uses some derived data types: sequences, strings, and
tuples. Sequences are lists filled with elements of the same data type. Strings
are sequences of characters. Tuples contain a list of values, potentially of
different types. In addition, \progname \ uses functions, which
are defined by the data types of their inputs and outputs. Local functions are
described by giving their type signature followed by their specification.

\section{Module Decomposition}

The following table is taken directly from the Module Guide document for this project.

\begin{table}[h!]
\centering
\begin{tabular}{p{0.3\textwidth} p{0.6\textwidth}}
\toprule
\textbf{Level 1} & \textbf{Level 2}\\
\midrule

{Hardware-Hiding} & ~ \\
\midrule

\multirow{7}{0.3\textwidth}{Behaviour-Hiding} & Input Parameters\\
& Output Format\\
& Output Verification\\
& Temperature ODEs\\
& Energy Equations\\ 
& Control Module\\
& Specification Parameters Module\\
\midrule

\multirow{3}{0.3\textwidth}{Software Decision} & {Sequence Data Structure}\\
& ODE Solver\\
& Plotting\\
\bottomrule

\end{tabular}
\caption{Module Hierarchy}
\label{TblMH}
\end{table}

\newpage
~\newpage

\section{MIS of \wss{Module Name}} \label{Module} 

\wss{You can reference SRS labels, such as R\ref{R_Inputs}.}

\wss{It is also possible to use \LaTeX for hypperlinks to external documents.}

\subsection{Module}

\wss{Short name for the module}

\subsection{Uses}


\subsection{Syntax}

\subsubsection{Exported Constants}

\subsubsection{Exported Access Programs}

\begin{center}
\begin{tabular}{p{2cm} p{4cm} p{4cm} p{2cm}}
\hline
\textbf{Name} & \textbf{In} & \textbf{Out} & \textbf{Exceptions} \\
\hline
\wss{accessProg} & - & - & - \\
\hline
\end{tabular}
\end{center}

\subsection{Semantics}

\subsubsection{State Variables}

\wss{Not all modules will have state variables.  State variables give the module
  a memory.}

\subsubsection{Environment Variables}

\wss{This section is not necessary for all modules.  Its purpose is to capture
  when the module has external interaction with the environment, such as for a
  device driver, screen interface, keyboard, file, etc.}

\subsubsection{Assumptions}

\wss{Try to minimize assumptions and anticipate programmer errors via
  exceptions, but for practical purposes assumptions are sometimes appropriate.}

\subsubsection{Access Routine Semantics}

\noindent \wss{accessProg}():
\begin{itemize}
\item transition: \wss{if appropriate} 
\item output: \wss{if appropriate} 
\item exception: \wss{if appropriate} 
\end{itemize}

\wss{A module without environment variables or state variables is unlikely to
  have a state transition.  In this case a state transition can only occur if
  the module is changing the state of another module.}

\wss{Modules rarely have both a transition and an output.  In most cases you
  will have one or the other.}

\subsubsection{Local Functions}

\wss{As appropriate} \wss{These functions are for the purpose of specification.
  They are not necessarily something that is going to be implemented
  explicitly.  Even if they are implemented, they are not exported; they only
  have local scope.}
\newpage

\section{MIS of Sensor Reading Module} \label{Module} 

\subsection{Module}

M1: Sensor Reading Module.

\subsection{Uses}
Used to gather data on user presence in shared space to determine when to take before and after pictures of the shared space.


\subsection{Syntax}

\subsubsection{Exported Constants}
motionPresent: boolean

\subsubsection{Exported Access Programs}

\begin{center}
\begin{tabular}{p{2cm} p{4cm} p{4cm} p{2cm}}
\hline
\textbf{Name} & \textbf{In} & \textbf{Out} & \textbf{Exceptions} \\
\hline
timeMotion & timesOfMotion & timeMotionDetected & - \\
detector & sensorData & motionPresent & - \\
insertTime & - & - & - \\
\hline
\end{tabular}
\end{center}

\subsection{Semantics}

\subsubsection{State Variables}

timesOfMotion: List[datetime] - List of time when motion was detected by the sensor.

\subsubsection{Environment Variables}
sensorData: sensorDataType - Data acquired by the sensor with data type supported by sensor.

\subsubsection{Access Routine Semantics}

\noindent timeMotion(timesOfMotion: datetime) -$>$ datetime:
\begin{itemize}
\item input: List of time stamps of when motion was detected. 
\item output: Last timestamp of when motion was detected. 
\end{itemize}

\noindent detector(sensorData: sensorDataType) -$>$ boolean:
\begin{itemize}
\item input: Data received from sensor. 
\item output: True of false value depending on if motion was detected. 
\end{itemize}

\subsubsection{Local Functions}

insertTime() - Inserting time stamp into timesOfMotion variable.


\newpage

\section{MIS of Image Capture Module} \label{Module} 

\subsection{Module}

M2: Image Capture Module.

\subsection{Uses}
Capture image of shared space using camera system.

\subsection{Syntax}

\subsubsection{Exported Constants}
image: png

\subsubsection{Exported Access Programs}

\begin{center}
\begin{tabular}{p{4cm} p{4cm} p{4cm} p{2cm}}
\hline
\textbf{Name} & \textbf{In} & \textbf{Out} & \textbf{Exceptions} \\
\hline
captureImage & - & image & - \\
\hline
\end{tabular}
\end{center}

\subsection{Semantics}

\subsubsection{Access Routine Semantics}

\noindent captureImage() -$>$ 'png':
\begin{itemize}
\item output: Image taken.
\end{itemize}

\subsubsection{Assumptions}
\begin{itemize}
\item captureImage will only be called when picture is needed to be taken, that is before user starts using shared space and five minutes after user has left shared space.
\end{itemize}


\newpage

\section{MIS of Image Upload Module} \label{Module} 

\subsection{Module}

M3: Image Upload Module.

\subsection{Uses}
Uploads the captured image to Raspberry Pi for the cleanliness detection system to use for analysis.

\subsection{Syntax}

\subsubsection{Exported Constants}
imgUploadErrorMessage : str - Message for cause of upload error.

\subsubsection{Exported Access Programs}

\begin{center}
\begin{tabular}{p{3cm} p{2cm} p{2cm} p{5cm}}
\hline
\textbf{Name} & \textbf{In} & \textbf{Out} & \textbf{Exceptions} \\
\hline
uploadImage & image & - & imgUploadErrorMessage \\
\hline
\end{tabular}
\end{center}

\subsection{Semantics}


\subsubsection{Access Routine Semantics}

\noindent uploadImage(image: png):
\begin{itemize}
\item input: Image taken from camera. 
\item exception: 
	\begin{itemize}
		\item ImageUploadInterrupted: Raised if error occurs during image upload.
	\end{itemize} 
\end{itemize}

\subsubsection{Assumptions}
\begin{itemize}
	\item Network is stable and never causes error in image upload.
	\item Power source is stable and never causes error in image upload.
\end{itemize}

\subsubsection{Local Functions}

captureImage() -$>$ png - Takes picture to be uploaded.

\newpage

\section{MIS of Request Listener Module} \label{Module} 


\subsection{Module}

M7: Request Listener Module.

\subsection{Uses}
Exposes cleanliness detector to camera and image by making it an application programming interface.

\subsection{Syntax}

\subsubsection{Exported Constants}

\subsubsection{Exported Access Programs}

\begin{center}
\begin{tabular}{p{2cm} p{4cm} p{4cm} p{2cm}}
\hline
\textbf{Name} & \textbf{In} & \textbf{Out} & \textbf{Exceptions} \\
\hline
getImages & - & images & - \\
\hline
\end{tabular}
\end{center}

\subsection{Semantics}


\subsubsection{Access Routine Semantics}

\noindent getImages() -$>$ List[png]:
\begin{itemize}
\item output: Two most recent images where one is the before and other is the after state of shared space after user is finished.
\end{itemize}

\newpage

\section{MIS of PreprocessingModule} \label{Module} 

\subsection{Module}

PreprocessingModule

\subsection{Uses}
The Preprocessing Module is used for preparing raw images (i.e. from the Image Upload module) to be submitted to subsequent modules (i.e. Object Detection and Scoring). A series of transformations is applied to the image.

\subsection{Syntax}

\subsubsection{Exported Constants}

\begin{itemize}
  \item SUPPORTED{\_}FORMATS: List[str] = ["JPEG", "PNG"]
  \item DEFAULT{\_}TRANSFORMS: torchvision.transforms.Compose\: A default set of PyTorch transformations, including resizing, normalization, and optional filtering
\end{itemize}

\subsubsection{Exported Access Programs}

\begin{center}
\begin{tabular}{p{5cm} p{3cm} p{3cm} p{2cm}}
\hline
\textbf{Name} & \textbf{In} & \textbf{Out} & \textbf{Exceptions} \\
\hline
process{\_}image & image: Image & Tensor & - \\
\hline
\end{tabular}
\end{center}

\subsection{Semantics}

\subsubsection{State Variables}
None



\subsubsection{Environment Variables}
\begin{itemize}
  \item ImageUploadModule: Module from which input image is recieved
\end{itemize}

\subsubsection{Assumptions}
\begin{itemize}
  \item The input image is in a valid format supported by PyTorch (e.g. JPEG, PNG)
  \item PyTorch and torchvision libraries are installed and functional
  \item Network connectivity is stable for receiving images through the cloud network
\end{itemize}


\subsubsection{Access Routine Semantics}

\noindent process{\_}image(image: Image):
\begin{itemize}
\item transition: Applies a sequence of PyTorch transformations to the input image 
\item output: Returns the transformed image as a Pytorch tensor, ready for subsequent object detection
\item exception: 
\begin{itemize}
  \item InvalidImageFormatException: Raised if the input image format is unsupported
  \item TransformationError: Raised if an error occurs during transformations
\end{itemize}

\end{itemize}
\noindent process{\_}image(image: Image):
\begin{itemize}
  \item transition: Applies pixel map transformations (e.g. resizing, cropping, filtering) to the input image
  \item output: Returns the transformed image
  \item exception: 
  \begin{itemize}
    \item TransformationError: Raised if an error occurs during transformations
  \end{itemize}
  
  \end{itemize}

\subsubsection{Local Functions}

\begin{itemize}
  \item validate{\_}image{\_}format(image: Image) $\rightarrow$ bool: Checks if the input image format is supported
  \item create{\_}transforms() $\rightarrow$ torchvision.transforms.Compose\: Creates a PyTorch Compose object for the default set of transformations
  \item apply{\_}transforms(image: Image, transforms: torchvision.transforms.Compose) $\rightarrow$ Tensor: Applies the given transformations to the input image
\end{itemize}

\newpage

\section{MIS of ObjectDetection} \label{Module} 

\subsection{Module}

ObjectDetectionModule

\subsection{Uses}
To detect objects in an input tensor (preprocessed image) using a pretrained object detector (e.g. Faster R-CNN ResNet-50 FPN model). The output is a list of detected objects, where each object is represented as a HouseObject.

\subsection{Syntax}

\subsubsection{Exported Constants}

\begin{itemize}
  \item MODEL{\_}NAME: str = "FasterRCNN{\_}ResNet50{\_}FPN"
  \item CONFIDENCE{\_}THRESHOLD: float = 0.5 (minimum confidence score for object detection results)
\end{itemize}

\subsubsection{Exported Access Programs}

\begin{center}
\begin{tabular}{p{5cm} p{3cm} p{3cm} p{2cm}}
\hline
\textbf{Name} & \textbf{In} & \textbf{Out} & \textbf{Exceptions} \\
\hline
detect{\_}objects & input{\_}tensor: Tensor & List[HouseObject] & - \\
\hline
\end{tabular}
\end{center}

\subsection{Semantics}

\subsubsection{State Variables}
\begin{itemize}
	\item self.model: torchvision.models.detection.FasterRCNN
\end{itemize}


\subsubsection{Environment Variables}
None

\subsubsection{Assumptions}
\begin{itemize}
  \item The input tensor is correctly preprocessed and normalized according to the requirements of the Faster R-CNN model
  \item PyTorch and torchvision libraries are installed and functional
\end{itemize}


\subsubsection{Access Routine Semantics}

\noindent detect{\_}objects
\begin{itemize}
\item transition: Uses the pretrained Faster R-CNN model to detect objects in the input tensor, filters results based on the confidence threshold
\item output: Returns a list of detected objects, where each object is represented as a HouseObject
\item exception: 
\begin{itemize}
  \item ModelNotLoadedException: Raised if the model fails to load
  \item DetectionError: Raised if an error occurs during detection process
\end{itemize}

\end{itemize}


\subsubsection{Local Functions}

\begin{itemize}
  \item load{\_}model() $\rightarrow$ torchvision.models.detection.FasterRCNN: Loads pretrained Faster R-CNN ResNet-50 FPN model
  \item filter{\_}detections(detections: List[Dict], threshold: float) $\rightarrow$ List[HouseObject]: Filters the raw detections based on the confidence threshold and converts them to HouseObject instances
\end{itemize}

\newpage

\section{MIS of Scoring} \label{Module} 

\subsection{Module}

ScoringModule

\subsection{Uses}
To evaluate changes in a room by comparing two sets of detected objects (representing "before" and "after" states) and assign a cleanliness score based on the differences

\subsection{Syntax}

\subsubsection{Exported Constants}

\begin{itemize}
  \item MAX{\_}SCORE: int = 100 (maximum cleanliness score)
  \item MIN{\_}SCORE: int = 0 (minimum cleanliness score)
  \item OBJECT{\_}WEIGHTS: dict (Weights for scoring different objects based on their categories)
\end{itemize}

\subsubsection{Exported Access Programs}

\begin{center}
\begin{tabular}{p{5cm} p{3cm} p{3cm} p{2cm}}
\hline
\textbf{Name} & \textbf{In} & \textbf{Out} & \textbf{Exceptions} \\
\hline
calculate{\_}cleanliness & before: List[HouseObject], after: List[HouseObject] & int & - \\
\hline
\end{tabular}
\end{center}

\subsection{Semantics}

\subsubsection{State Variables}
None


\subsubsection{Environment Variables}
None

\subsubsection{Assumptions}
\begin{itemize}
  \item Input lists 'before' and 'after' contain valid HouseObject instances
  \item The HouseObject class provides sufficient information (e.g., label, confidence, bounding box) to compare objects between the two states
\end{itemize}


\subsubsection{Access Routine Semantics}

\noindent calculate{\_}cleanliness
\begin{itemize}
\item transition: Compares the before and after lists to identify added, removed, or moved objects; computes a cleanliness score based on these changes
\item output: Returns an integer cleanliness score between MIN{\_}SCORE and MAX{\_}SCORE
\item exception: 
\begin{itemize}
  \item ModelNotLoadedException: Raised if the model fails to load
  \item DetectionError: ScoringError: Raised if an error occurs during the scoring process
\end{itemize}

\end{itemize}


\subsubsection{Local Functions}

\begin{itemize}
  \item compare{\_}objects(before: List[HouseObject], after: List[HouseObject]) $\rightarrow$ Dict[str, List[HouseObject]]: Identifies added, removed, and moved objects between the two lists
  \item compute{\_}score(changes: Dict[str, List[HouseObject]]) $\rightarrow$ int: Computes the cleanliness score based on detected changes
\end{itemize}

\newpage

\bibliographystyle {plainnat}
\bibliography {../../../refs/References}


\newpage

\section{Appendix} \label{Appendix}

\wss{Extra information if required}

\newpage{}

\section*{Appendix --- Reflection}

\wss{Not required for CAS 741 projects}

The information in this section will be used to evaluate the team members on the
graduate attribute of Problem Analysis and Design.

\input{../../Reflection.tex}

\begin{enumerate}
  \item What went well while writing this deliverable? 
  \item What pain points did you experience during this deliverable, and how
    did you resolve them?
  \item Which of your design decisions stemmed from speaking to your client(s)
  or a proxy (e.g. your peers, stakeholders, potential users)? For those that
  were not, why, and where did they come from?
  \item While creating the design doc, what parts of your other documents (e.g.
  requirements, hazard analysis, etc), it any, needed to be changed, and why?
  \item What are the limitations of your solution?  Put another way, given
  unlimited resources, what could you do to make the project better? (LO\_ProbSolutions)
  \item Give a brief overview of other design solutions you considered.  What
  are the benefits and tradeoffs of those other designs compared with the chosen
  design?  From all the potential options, why did you select the documented design?
  (LO\_Explores)
\end{enumerate}


\end{document}