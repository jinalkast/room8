\documentclass{article}

\usepackage{float}
\restylefloat{table}

\usepackage{booktabs}

\title{Team Contributions: Rev 0\\\progname}

\author{\authname}

\date{}

\input{../Comments}
%% Common Parts

\newcommand{\progname}{Room8} % PUT YOUR PROGRAM NAME HERE
\newcommand{\authname}{Team 19
\\ Mohammed Abed
\\ Maged Armanios
\\ Jinal Kasturiarachchi
\\ Jane Klavir
\\ Harshil Patel} % AUTHOR NAMES

\usepackage{hyperref}
    \hypersetup{colorlinks=true, linkcolor=blue, citecolor=blue, filecolor=blue,
                urlcolor=blue, unicode=false}
    \urlstyle{same}
                                


\begin{document}

\maketitle

This document summarizes the contributions of each team member for the Rev 0
Demo.  The time period of interest is the time between the POC demo and the Rev
0 demo.

\section{Demo Plans}
We aim to demonstrate a more polished version of the frontend application that has some integration with the cleanliness management system. In the demo, the camera system will most likely have not been implemented yet and will be substituted with captured images by the team.

\section{Team Meeting Attendance}
\begin{table}[H]
\centering
\begin{tabular}{ll}
\toprule
\textbf{Student} & \textbf{Meetings}\\
\midrule
Total & 12\\
Mohammed Abed & 9\\
Maged Armanios & 9\\
Jinal Kasturiarachchi & 9\\
Jane Klavir & 11\\
Harshil Patel & 11\\
\bottomrule
\end{tabular}
\end{table}

\textbf{Note}: Currently, the team is divided into two sub-teams which is why the numbers look a little strange but it's just because each sub-team meets separately sometimes.

\section{Supervisor/Stakeholder Meeting Attendance}


\begin{table}[H]
\centering
\begin{tabular}{ll}
\toprule
\textbf{Student} & \textbf{Meetings}\\
\midrule
Total & 4\\
Mohammed Abed & 4\\
Maged Armanios & 4\\
Jinal Kasturiarachchi & 3\\
Jane Klavir & 4\\
Harshil Patel & 3\\
\bottomrule
\end{tabular}
\end{table}


\section{Lecture Attendance}

\begin{table}[H]
\centering
\begin{tabular}{ll}
\toprule
\textbf{Student} & \textbf{Lectures}\\
\midrule
Total & 11\\
Mohammed Abed & 1\\
Maged Armanios & 2\\
Jinal Kasturiarachchi & 1\\
Jane Klavir & 2\\
Harshil Patel & 1\\
\bottomrule
\end{tabular}
\end{table}


\section{TA Document Discussion Attendance}

\begin{table}[H]
\centering
\begin{tabular}{ll}
\toprule
\textbf{Student} & \textbf{Lectures}\\
\midrule
Total & 4\\
Mohammed Abed & 3\\
Maged Armanios & 4\\
Jinal Kasturiarachchi & 3\\
Jane Klavir & 4\\
Harshil Patel & 3\\
\bottomrule
\end{tabular}
\end{table}

\section{Commits}

\begin{table}[H]
\centering
\begin{tabular}{lll}
\toprule
\textbf{Student} & \textbf{Commits} & \textbf{Percent}\\
\midrule
Total & 360 & 100\% \\
Mohammed Abed & 56 & 15.6\% \\
Maged Armanios & 133 & 36.9\% \\
Jinal Kasturiarachchi & 55 & 15.3\% \\
Jane Klavir & 59 & 16.4\% \\
Harshil Patel & 57 & 15.8\% \\
\bottomrule
\end{tabular}
\end{table}

\section{Issue Tracker}

\begin{table}[H]
\centering
\begin{tabular}{lll}
\toprule
\textbf{Student} & \textbf{Authored (O+C)} & \textbf{Assigned (C only)}\\
\midrule
Mohammed Abed & 15 & 4 \\
Maged Armanios & 39 & 3 \\
Jinal Kasturiarachchi & 16 & 0 \\
Jane Klavir & 4 & 1 \\
Harshil Patel & 2 & 2 \\
\bottomrule
\end{tabular}
\end{table}

\textbf{Note}: While the issue tracker is used frequently and in our opinion efficiently, we do not assign tickets as much as we probably should.

\section{CICD}

As stated in the development plan, CICD will be used to automate testing and deployment of changes in a "dev" branch. After validating the "dev" branch, the deployment of the "main" branch will be completed manually on an as-needed-basis.
In addition to that, CI/CD wil also be used to automate the compilation of latex documents, which contain almost all relevent project documentioon.
\end{document}