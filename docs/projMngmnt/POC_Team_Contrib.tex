\documentclass{article}

\usepackage{float}
\restylefloat{table}

\usepackage{booktabs}

\title{Team Contributions: POC\\\progname}

\author{\authname}

\date{}

\input{../Comments}
%% Common Parts

\newcommand{\progname}{Room8} % PUT YOUR PROGRAM NAME HERE
\newcommand{\authname}{Team 19
\\ Mohammed Abed
\\ Maged Armanios
\\ Jinal Kasturiarachchi
\\ Jane Klavir
\\ Harshil Patel} % AUTHOR NAMES

\usepackage{hyperref}
    \hypersetup{colorlinks=true, linkcolor=blue, citecolor=blue, filecolor=blue,
                urlcolor=blue, unicode=false}
    \urlstyle{same}
                                


\begin{document}

\maketitle

This document summarizes the contributions of each team member up to the POC
Demo.  The time period of interest is the time between the beginning of the term
and the POC demo.

\section{Demo Plans}

\wss{What will you be demonstrating}
We will be demonstrating a proof of concept of our project Room8. The demo aims to demonstrate a proof of concept of our cleanliness detection system which will determine a difference in cleanliness between two images from the same reference points alongside it's accompanying application. For the POC demo, the algorithm will not be receiving input from a camera system like specified in the project requirements, nor will the cleanliness detection system be connected to the app.
\section{Team Meeting Attendance}

\wss{For each team member how many team meetings have they attended over the
time period of interest.  This number should be determined from the meeting
issues in the team's repo.  The first entry in the table should be the total
number of team meetings held by the team.}

\begin{table}[H]
\centering
\begin{tabular}{ll}
\toprule
\textbf{Student} & \textbf{Meetings}\\
\midrule
Total & 7\\
Mohammed Abed & 7\\
Maged Armanios & 7\\
Jinal Kasturiarachchi & 7\\
Jane Klavir & 7\\
Harshil Patel & 7\\
\bottomrule
\end{tabular}
\end{table}

\wss{If needed, an explanation for the counts can be provided here.}

\section{Supervisor/Stakeholder Meeting Attendance}

\wss{For each team member how many supervisor/stakeholder team meetings have
they attended over the time period of interest.  This number should be determined
from the supervisor meeting issues in the team's repo.  The first entry in the
table should be the total number of supervisor and team meetings held by the
team.  If there is no supervisor, there will usually be meetings with
stakeholders (potential users) that can serve a similar purpose.}

\begin{table}[H]
\centering
\begin{tabular}{ll}
\toprule
\textbf{Student} & \textbf{Meetings}\\
\midrule
Total & 2\\
Mohammed Abed & 2\\
Maged Armanios & 2\\
Jinal Kasturiarachchi & 1\\
Jane Klavir & 2\\
Harshil Patel & 1\\
\bottomrule
\end{tabular}
\end{table}

\wss{If needed, an explanation for the counts can be provided here.}

\section{Lecture Attendance}

\wss{For each team member how many lectures have they attended over the time
period of interest.  This number should be determined from the lecture issues in
the team's repo.  The first entry in the table should be the total number of
lectures since the beginning of the term.}

\begin{table}[H]
\centering
\begin{tabular}{ll}
\toprule
\textbf{Student} & \textbf{Lectures}\\
\midrule
Total & 11\\
Mohammed Abed & 1\\
Maged Armanios & 2\\
Jinal Kasturiarachchi & 1\\
Jane Klavir & 2\\
Harshil Patel & 1\\
\bottomrule
\end{tabular}
\end{table}

\wss{If needed, an explanation for the lecture attendance can be provided here.}

\section{TA Document Discussion Attendance}

\wss{For each team member how many of the informal document discussion meetings
with the TA were attended over the time period of interest.}

\begin{table}[H]
\centering
\begin{tabular}{ll}
\toprule
\textbf{Student} & \textbf{Lectures}\\
\midrule
Total & 3\\
Mohammed Abed & 3\\
Maged Armanios & 3\\
Jinal Kasturiarachchi & 3\\
Jane Klavir & 3\\
Harshil Patel & 3\\
\bottomrule
\end{tabular}
\end{table}

\wss{If needed, an explanation for the attendance can be provided here.}

\section{Commits}

\wss{For each team member how many commits to the main branch have been made
over the time period of interest.  The total is the total number of commits for
the entire team since the beginning of the term.  The percentage is the
percentage of the total commits made by each team member.}

\begin{table}[H]
\centering
\begin{tabular}{lll}
\toprule
\textbf{Student} & \textbf{Commits} & \textbf{Percent}\\
\midrule
Total & 144 & 100\% \\
Mohammed Abed & 31 & 21.5\% \\
Maged Armanios & 48 & 33.3\% \\
Jinal Kasturiarachchi & 14 & 9.7\% \\
Jane Klavir & 22 & 15.3\% \\
Harshil Patel & 29 & 20.1\% \\
\bottomrule
\end{tabular}
\end{table}

\wss{If needed, an explanation for the counts can be provided here.  For
instance, if a team member has more commits to unmerged branches, these numbers
can be provided here.  If multiple people contribute to a commit, git allows for
multi-author commits.}

\section{Issue Tracker}

\wss{For each team member how many issues have they authored (including open and
closed issues (O+C)) and how many have they been assigned (only counting closed
issues (C only)) over the time period of interest.}

\begin{table}[H]
\centering
\begin{tabular}{lll}
\toprule
\textbf{Student} & \textbf{Authored (O+C)} & \textbf{Assigned (C only)}\\
\midrule
Mohammed Abed & 4 & 0 \\
Maged Armanios & 19 & 13 \\
Jinal Kasturiarachchi & 4 & 0 \\
Jane Klavir & 5 & 0 \\
Harshil Patel & 2 & 0 \\
\bottomrule
\end{tabular}
\end{table}

\wss{If needed, an explanation for the counts can be provided here.}\\
Includes peer review issues opened.
\section{CICD}

\wss{Say how CICD will be used in your project}
\wss{If your team has additional metrics of productivity, please feel free to
add them to this report.}
As stated in the development plan, CICD will be used to automate testing and deployment of changes in a "dev" branch. After validating the "dev" branch, the deployment of the "main" branch will be completed manually on an as-needed-basis.



\end{document}