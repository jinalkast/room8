\documentclass{article}

\usepackage{booktabs}
\usepackage{tabularx}
\usepackage{hyperref}

\hypersetup{
    colorlinks=true,       % false: boxed links; true: colored links
    linkcolor=red,          % color of internal links (change box color with linkbordercolor)
    citecolor=green,        % color of links to bibliography
    filecolor=magenta,      % color of file links
    urlcolor=cyan           % color of external links
}

\title{Hazard Analysis\\Room8}

\author{Mohammed Abed \\ 
        Maged Armanios\\
        Jinal Kasturiarachchi\\
        Jane Klavir\\
        Harshil Patel\\}

\date{2024-10-22}

\input{../Comments}
%% Common Parts

\newcommand{\progname}{Room8} % PUT YOUR PROGRAM NAME HERE
\newcommand{\authname}{Team 19
\\ Mohammed Abed
\\ Maged Armanios
\\ Jinal Kasturiarachchi
\\ Jane Klavir
\\ Harshil Patel} % AUTHOR NAMES

\usepackage{hyperref}
    \hypersetup{colorlinks=true, linkcolor=blue, citecolor=blue, filecolor=blue,
                urlcolor=blue, unicode=false}
    \urlstyle{same}
                                


\begin{document}

\maketitle
\thispagestyle{empty}

~\newpage

\pagenumbering{roman}

\begin{table}[hp]
\caption{Revision History} \label{TblRevisionHistory}
\begin{tabularx}{\textwidth}{llX}
\toprule
\textbf{Date} & \textbf{Developer(s)} & \textbf{Change}\\
\midrule
Date1 & Name(s) & Description of changes\\
Date2 & Name(s) & Description of changes\\
... & ... & ...\\
\bottomrule
\end{tabularx}
\end{table}

~\newpage

\tableofcontents

~\newpage

\pagenumbering{arabic}

\wss{You are free to modify this template.}

\section{Introduction}

\wss{You can include your definition of what a hazard is here.}\\
Room8 is a suite of tools aimed at reducing the occurrence of frustrating situations between roommates. Room8 is expected to be implemented as a mobile application and interact with the physical world using a camera and as a result, is expected to handle sensitive user data such as addresses, names, birthdays, images, and financial details. This document aims to outline the scope, critical assumptions, potential failures, and mitigation strategies for Room8. Hazards in the system can be caused by data privacy issues, system malfunction / misuse, and legal / compliance issues.


\section{Scope and Purpose of Hazard Analysis}

\wss{You should say what \textbf{loss} could be incurred because of the
hazards.}\\
The purpose of this hazard analysis is to identify, be aware, and mitigate losses that can be incurred as a result of hazards in the system. There a multiple ways to mitigate the losses such as following appropriate regulations, implementing thorough testing, and informing users how to properly use the system. By examining as many scenarios as possible where the system can cause harm and recording it in this document, the development team aims to minimize the harm dealt to users, stakeholders, and development team. Possible loss that can occur from hazards includes financial loss, loss of reputation, and service disruptions.   

\section{System Boundaries and Components}
This section goes over the components that the system can be divided into.

\subsection{User Smart Phone}
Smart phone the user is using with the supported version of Android or iOS.

\subsection{Camera}
Responsible for taking picture for cleanliness detection analysis when sensor sends information of user.

\subsection{Motion Sensor}
Detects movement in the shared space to determine if user has entered or exited shared space signaling the camera for a picture.

\subsection{Smart Phone Application}
A mobile application installed on smart phones which have versions of Android and iOS that is currently being supported mobile providers. This includes front-end of the system where users can see details and change settings of various back-end components listed below.


\subsection{Authentication}
Authentication using OAuth of user credentials and house details are processed in this component including the update of information mentioned previously.

\subsection{SMS ChatBot}
ChatBot responsible for sending messages to group chat of home members for notifying them of cleanliness assessment, expenses from bill splitter, or reminders to complete tasks.

\subsection{Calendar Tool}
Allows users to add events to calendar and display to other housemates, if involved in event, in their respective calendars. Also houses logic for generating chore/cleaning schedule and adding in calendars of users.

\subsection{Cleanliness Manager}
Responsible for handling and showing users information about their cleanliness after use of shared space and history of events in the space.

\subsection{Bill Splitter}
Calculate charges due from a shared expense and keeps track of which expenses are due from each user and who they owe using the SMS ChatBot to notify users. Also stores history of expenses and charges paid.

\subsection{Database}
Used to securely store user and house information, calendar events, expense history, and pictures for cleanliness calculator.

\subsection{Cleanliness Calculator}
Runs algorithm for detecting change in environment through input received from hardware and stores information for the user to view on application.



\section{Critical Assumptions}

\wss{These assumptions that are made about the software or system.  You should
minimize the number of assumptions that remove potential hazards.  For instance,
you could assume a part will never fail, but it is generally better to include
this potential failure mode.}

\section{Failure Mode and Effect Analysis}

\wss{Include your FMEA table here. This is the most important part of this document.}
\wss{The safety requirements in the table do not have to have the prefix SR.
The most important thing is to show traceability to your SRS. You might trace to
requirements you have already written, or you might need to add new
requirements.}
\wss{If no safety requirement can be devised, other mitigation strategies can be
entered in the table, including strategies involving providing additional
documentation, and/or test cases.}

\section{Safety and Security Requirements}

\wss{Newly discovered requirements.  These should also be added to the SRS.  (A
rationale design process how and why to fake it.)}

\section{Roadmap}

\wss{Which safety requirements will be implemented as part of the capstone timeline?
Which requirements will be implemented in the future?}

\newpage{}

\section*{Appendix --- Reflection}

\wss{Not required for CAS 741}

\input{../Reflection.tex}

\begin{enumerate}
    \item What went well while writing this deliverable? 
    \item What pain points did you experience during this deliverable, and how
    did you resolve them?
    \item Which of your listed risks had your team thought of before this
    deliverable, and which did you think of while doing this deliverable? For
    the latter ones (ones you thought of while doing the Hazard Analysis), how
    did they come about?
    \item Other than the risk of physical harm (some projects may not have any
    appreciable risks of this form), list at least 2 other types of risk in
    software products. Why are they important to consider?
\end{enumerate}

\end{document}