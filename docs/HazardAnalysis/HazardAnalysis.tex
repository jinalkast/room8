\documentclass{article}

\usepackage{booktabs}
\usepackage{tabularx}
\usepackage{hyperref}
\usepackage{adjustbox}
\usepackage{longtable}
\usepackage{float}
\usepackage{enumerate}
\usepackage{multirow}
\usepackage{array}
\usepackage[margin=35mm,paper=a4paper]{geometry}


\hypersetup{
    colorlinks=true,       % false: boxed links; true: colored links
    linkcolor=red,          % color of internal links (change box color with linkbordercolor)
    citecolor=green,        % color of links to bibliography
    filecolor=magenta,      % color of file links
    urlcolor=cyan           % color of external links
}

\title{Hazard Analysis\\Room8}

\author{Mohammed Abed \\ 
        Maged Armanios\\
        Jinal Kasturiarachchi\\
        Jane Klavir\\
        Harshil Patel\\}

\date{2024-10-25}

\input{../Comments}
%% Common Parts

\newcommand{\progname}{Room8} % PUT YOUR PROGRAM NAME HERE
\newcommand{\authname}{Team 19
\\ Mohammed Abed
\\ Maged Armanios
\\ Jinal Kasturiarachchi
\\ Jane Klavir
\\ Harshil Patel} % AUTHOR NAMES

\usepackage{hyperref}
    \hypersetup{colorlinks=true, linkcolor=blue, citecolor=blue, filecolor=blue,
                urlcolor=blue, unicode=false}
    \urlstyle{same}
                                


\def\changemargin#1#2{\list{}{\rightmargin#2\leftmargin#1}\item[]}
\let\endchangemargin=\endlist 

\begin{document}

\maketitle
\thispagestyle{empty}

~\newpage

\pagenumbering{roman}

\begin{table}[hp]
\caption{Revision History} \label{TblRevisionHistory}
\begin{tabularx}{\textwidth}{llX}
\toprule
\textbf{Date} & \textbf{Developer(s)} & \textbf{Change}\\
\midrule
Date1 & Name(s) & Description of changes\\
Date2 & Name(s) & Description of changes\\
... & ... & ...\\
\bottomrule
\end{tabularx}
\end{table}

~\newpage

\tableofcontents

~\newpage

\pagenumbering{arabic}

\wss{You are free to modify this template.}

\section{Introduction}

\wss{You can include your definition of what a hazard is here.}\\
Room8 is a suite of tools aimed at reducing the occurrence of frustrating situations between roommates. Room8 is expected to be implemented as a mobile application and interact with the physical world using a camera and as a result, is expected to handle sensitive user data such as addresses, names, birthdays, images, and financial details. This document aims to outline the scope, critical assumptions, potential failures, and mitigation strategies for Room8. Hazards in the system can be caused by data privacy issues, system malfunction / misuse, and legal / compliance issues.


\section{Scope and Purpose of Hazard Analysis}

\wss{You should say what \textbf{loss} could be incurred because of the
hazards.}\\
The purpose of this hazard analysis is to identify, be aware, and mitigate losses that can be incurred as a result of hazards in the system. There a multiple ways to mitigate the losses such as following appropriate regulations, implementing thorough testing, and informing users how to properly use the system. By examining as many scenarios as possible where the system can cause harm and recording it in this document, the development team aims to minimize the harm dealt to users, stakeholders, and development team. Possible loss that can occur from hazards includes financial loss, loss of reputation, and service disruptions.   

\section{System Boundaries and Components}
This section goes over the components that the system can be divided into.

\subsection{User Device}
Smart phone the user is using with the supported version of Android or iOS.

\subsection{Camera}
Responsible for taking picture for cleanliness detection analysis when sensor sends information of user.

\subsection{Motion Sensor}
Detects movement in the shared space to determine if user has entered or exited shared space signaling the camera for a picture.

\subsection{PWA Interface}
A mobile application installed on smart phones which have versions of Android and iOS that is currently being supported mobile providers. This includes front-end of the system where users can see details and change settings of various back-end components listed below.

\subsection{Authentication}
Authentication using OAuth of user credentials and house details are processed in this component including the update of information mentioned previously.

\subsection{SMS ChatBot}
ChatBot responsible for sending messages to group chat of home members for notifying them of cleanliness assessment, expenses from bill splitter, or reminders to complete tasks.

\subsection{Calendar Tool}
Allows users to add events to calendar and display to other housemates, if involved in event, in their respective calendars. Also houses logic for generating chore/cleaning schedule and adding in calendars of users.

\subsection{Cleanliness Manager}
Runs algorithm for detecting change in environment through input received from hardware and stores user's information for the user to view on application along with history of cleanliness.

\subsection{Bill Splitter}
Calculate charges due from a shared expense and keeps track of which expenses are due from each user and who they owe using the SMS ChatBot to notify users. Also stores history of expenses and charges paid for user to view.

\subsection{Database}
Used to securely store user and house information, calendar events, expense history, and pictures for cleanliness calculator.


\section{Critical Assumptions}

\wss{These assumptions that are made about the software or system.  You should
minimize the number of assumptions that remove potential hazards.  For instance,
you could assume a part will never fail, but it is generally better to include
this potential failure mode.}
\begin{itemize}
\item \textbf{CA1}: Homes will have a consistent and uninterrupted supply of electricity available.
\item \textbf{CA2}: Homes will have internet speeds capable of streaming video.
\item \textbf{CA3}: Every resident of a shared home will have their own personal electronic device.
\item \textbf{CA4}: Users have used other applications before and are familiar with common signifiers, mappings, and UI metaphores (ex. Heart implies like).
\item \textbf{CA5}: External services, such as location services, map integrations, and calendar APIs will be available and reliable.
\item \textbf{CA6}: Users' devices will have additional free storage beyond the what's required for the applications install.
\item \textbf{CA7}: Camera setup in shared environment will not be moved or blocked to ensure clear pictures of space.
\end{itemize}

\newpage

\newgeometry{left=5mm,right=5mm}

\section{Failure Mode and Effect Analysis}

\wss{Include your FMEA table here. This is the most important part of this document.}
\wss{The safety requirements in the table do not have to have the prefix SR.
The most important thing is to show traceability to your SRS. You might trace to
requirements you have already written, or you might need to add new
requirements.}
\wss{If no safety requirement can be devised, other mitigation strategies can be
entered in the table, including strategies involving providing additional
documentation, and/or test cases.}\\

\begin{longtable}{|>{\raggedright\arraybackslash}p{0.13\linewidth} | >{\raggedright\arraybackslash}p{0.13\linewidth} | >{\raggedright\arraybackslash}p{0.13\linewidth}| >{\raggedright\arraybackslash}p{0.13\linewidth}| >{\raggedright\arraybackslash}p{0.13\linewidth}| >{\raggedright\arraybackslash}p{0.05\linewidth}| >{\raggedright\arraybackslash}p{0.05\linewidth}| >{\raggedright\arraybackslash}p{0.07\linewidth}|}
    \caption{\bf FMEA Table} \label{tab:my_label} \\
    
    \hline
    \textbf{Design Functions} & \textbf{Failure Modes} & \textbf{Effects of Failure} & \textbf{Causes of Failure} & \textbf{Recommended Action} & \textbf{SR} & \textbf{Ref} & \textbf{Severity}\\
    \hline
    \endfirsthead
    
    \hline
    \textbf{Design Functions} & \textbf{Failure Modes} & \textbf{Effects of Failure} & \textbf{Causes of Failure} & \textbf{Recommended Action} & \textbf{SR} & \textbf{Ref} & \textbf{Severity}\\
    \hline
    \endhead
    
    \hline
    \endfoot
    
    \hline
    \endlastfoot
    
    User is setting up camera & ??? \newline & ???\newline & ???  \newline & ??? \newline???  \newline & SR?  \newline & H1.1  \newline & BLANK\\


    \hline
    Camera Takes Picture & Something irregular occuring in-frame \newline & Cleanliness detection algorithm does not produce good results due to bad input\newline & a. Camera field of view being blocked \newline b. Problem with lighting  \newline & a. Take hourly pictures (i.e. as long as motion not detected), use last picture taken \newline b. Same as H2.1a  \newline & SR?  \newline & H2.1  \newline & BLANK\\
    & Delay in camera shot timing & User in-frame, user privacy is breached & a. Motion sensor not properly connected to camera  \newline b. Motion sensor not working &  a. Remove picture from database, follow H2.1a, do troubleshooting for motion sensor \newline b. Same as H2.2a \newline & ??? & H2.2 & BLANK\\
    
    \hline
    Motion Sensor Detects Motion  & Motion should not be detected (false positive) \newline  & Unnecessary computation and notifications \newline & a. Motion sensor is overly sensitive \newline b. Some brief/light movement that should not be classified as motion occurred (e.g., insect flew by) \newline c. Continuous movement that should not be classified as motion is occurring (e.g., air conditioner causing curtain to move) \newline  & a. Include regular motion-calibration testing in system design \newline b. Have a clause in the cleanliness detection algorithm to check for false positives (i.e., if the before/after pictures do not meet a threshold to be considered dissimilar) \newline c. ???? & SR?, SR? & H3.1 & Medium\\
    & Motion is not detected although it is occurring (false negative) \newline & Cleanliness detection does not occur \newline & a. Motion sensor is underly sensitive \newline b. Motion sensor is broken \newline &  a. Same as H3.1a \newline b. Same as H3.1a \newline & ??? & H3.2 & BLANK\\
    
    \hline
    System Authenticates User & Bad actor logs in to a user's account& Bad actor alters account credentials. Sensitive user information is revealed, such as full name and address.\newline\newline Unauthorized use of services occurs.\newline\newline Account lockout.\newline\newline Bad actor impersonates user, allowing them to disrupt services for other users within the same house. & Data breach.\newline\newline Weak account password.\newline\newline Lack of multi-factor authentication.\newline\newline Outdated/insecure methods of storing user credentials.\newline\newline Man-in-the-middle attacks.\newline\newline Brute force attacks. & Check for unusual login patterns, such as different geolocations, IP addresses, and repeated failed attempts on the same login.\newline\newline Require all account passwords to satisfy a minimum password strength criterion.\newline\newline Impose rate limits on failed authentication attempts.\newline\newline Recommend multi-factor authentication to users. & SR?, SR? & H4.1 & Medium\\
    
    \hline
    ChatBot Sends Notifaction & User is notified of deleted event & False information sent to user and user gets annoyed& a. ChatBot SMS did not get update from calendar that event was deleted and user does not need to be informed & a. Have test cases covering testing if ChatBot SMS is updated when calendar events update & NFR222 & H5.1 & Medium\\
    
    \hline
    Scheduling an Event in the Calendar & Event cannot be scheduled & User is frustrated, and important information is not being sent to roommates &  a. Conflict occurred due to multiple users scheduling events simultaneously &  a. Put a lock on the calendar resource & SR?, SR? & 6.1 & Medium\\
    
    \hline
    Cleanliness Detection System Detects Changes in the Cleanliness of a Room & System falsely concludes that a room has become more dirty.\newline\newline System falsely concludes that a room has become more clean.\newline\newline System concludes that there are no changes despite there being changes.\newline\newline The system concludes that there are changes when there are no changes. & Conflict amongst roommates.\newline\newline Trust in the system declines. & Obstructions in the images captured by the camera.\newline\newline Improper calibration and timing of the motion sensor.\newline\newline Object detection algorithm has errors and classifies items incorrectly in an image.& Create base case tests for the cleanliness detection system, including no change, increase, decrease, and no room state change cases.\newline\newline Alert and instruct users to clear camera obstructions before setting up the system. & SR?, SR? & H7.1 & Low\\

    \hline
    Inputting into Bill Splitter & Amount owing not accurate & Users receiving false information & a. User made an error inputting bill \newline b. The bill changed (i.e. amount or people owing) \newline c. Bad actor creating false bills & a. Provide users with a mechanism to edit outstanding bills \newline b. Same as 8.1a \newline c. System has a way for roommates to delete bills and report misuse & SR?, SR? & H8.1 & Medium\\
    
    \hline
    Database something & NEW & NEW & NEW & NEW & SR?, SR? & H?? & Medium\\
    
\end{longtable}
\restoregeometry 

\newpage

\section{Safety and Security Requirements}

\wss{Newly discovered requirements.  These should also be added to the SRS.  (A
rationale design process how and why to fake it.)}

\section{Roadmap}

\wss{Which safety requirements will be implemented as part of the capstone timeline?
Which requirements will be implemented in the future?}

\newpage{}

\section*{Appendix --- Reflection}

\wss{Not required for CAS 741}

\input{../Reflection.tex}

\begin{enumerate}
    \item What went well while writing this deliverable? 
    \item What pain points did you experience during this deliverable, and how
    did you resolve them?
    \item Which of your listed risks had your team thought of before this
    deliverable, and which did you think of while doing this deliverable? For
    the latter ones (ones you thought of while doing the Hazard Analysis), how
    did they come about?
    \item Other than the risk of physical harm (some projects may not have any
    appreciable risks of this form), list at least 2 other types of risk in
    software products. Why are they important to consider?
\end{enumerate}

\end{document}