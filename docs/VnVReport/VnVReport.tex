\documentclass[12pt, titlepage]{article}

\usepackage{booktabs}
\usepackage{tabularx}
\usepackage{hyperref}
\hypersetup{
    colorlinks,
    citecolor=black,
    filecolor=black,
    linkcolor=red,
    urlcolor=blue
}
\usepackage[round]{natbib}
\usepackage{longtable}
\usepackage[shortlabels]{enumitem}
\usepackage{hyperref}
\hypersetup{
    colorlinks,
    citecolor=blue,
    filecolor=black,
    linkcolor=red,
    urlcolor=blue
}
\usepackage{amsmath}
\usepackage{array}
\usepackage{graphicx}
\usepackage[paper=a4paper]{geometry}

\input{../Comments}
%% Common Parts

\newcommand{\progname}{Room8} % PUT YOUR PROGRAM NAME HERE
\newcommand{\authname}{Team 19
\\ Mohammed Abed
\\ Maged Armanios
\\ Jinal Kasturiarachchi
\\ Jane Klavir
\\ Harshil Patel} % AUTHOR NAMES

\usepackage{hyperref}
    \hypersetup{colorlinks=true, linkcolor=blue, citecolor=blue, filecolor=blue,
                urlcolor=blue, unicode=false}
    \urlstyle{same}
                                


\begin{document}

\title{Verification and Validation Report: Room8} 
\author{\authname}
\date{\today}
	
\maketitle

\pagenumbering{roman}

\section{Revision History}

\begin{tabularx}{\textwidth}{p{3cm}p{2cm}X}
\toprule {\bf Date} & {\bf Version} & {\bf Notes}\\
\midrule
Date 1 & 1.0 & Notes\\
Date 2 & 1.1 & Notes\\
\bottomrule
\end{tabularx}

~\newpage

\section{Symbols, Abbreviations and Acronyms}

\renewcommand{\arraystretch}{1.2}
\begin{tabular}{l l} 
  \toprule		
  \textbf{Acronym} & \textbf{description}\\
  \midrule 
  SRS & Software Requirements Specification\\
  VnV & Verification and Validation\\
  CI/CD & Continuous Integration and Continuous deployment\\   
  API & Application Programming Interface\\
  Exp. & Expected\\
  Act. & Actual\\
  \bottomrule
\end{tabular}\\

\wss{symbols, abbreviations or acronyms -- you can reference the SRS tables if needed}

\newpage

\tableofcontents

\listoftables %if appropriate

\listoffigures %if appropriate

\newpage

\pagenumbering{arabic}

This document ...

\section{Functional Requirements Evaluation}

\subsection{FR211}
The system shall allow users to create an account using their Google account.
\renewcommand{\arraystretch}{1.5}
\begin{center}
  \testcase{
    \textbf{FST-UAHM-1} & \textbf{Profile Creation}\\\hline

    \textbf{Description} & Tests if the system is able to handle creating user 			profiles in the system from frontend inputs \\ 

    \small \textbf{Input} & Valid Google account sign in \\ 

    \small \textbf{Exp. Output} & Room8 account created and Frontend redirects user to the dashboard \\ 

    \small \textbf{Act. Output} & Room8 account created and Frontend redirects user to the dashboard\\ 
    
    \small \textbf{Result} & Pass\\ 

  }
\end{center}
\subsection{FR212}
The system shall allows users to log in to their account.
\renewcommand{\arraystretch}{1.5}
\begin{center}
  \testcase{
    \textbf{FST-UAHM-2} & \textbf{System Login}\\\hline

    \textbf{Description} & Tests if the system is able to recognize a user profile and to authenticate them to the system \\ 

    \textbf{Input} & Valid Google account matching the profile in the system \\ 

    \textbf{Exp. Output} & User profile is authenticated with a new session entry. Frontend redirects user to the dashboard \\ 

    \textbf{Act. Output} & User profile is authenticated with a new session entry. Frontend redirects user to the dashboard  \\ 
    
    \small \textbf{Result} & Pass\\ 
  }
\end{center}

\subsection{FR213}
The system shall allows users to log out of their account.
\renewcommand{\arraystretch}{1.5}
\begin{center}
  \testcase{
    \textbf{FST-UAHM-3} & \textbf{System Logout}\\\hline

    \textbf{Description} & Tests if the system logs the user out, invalidating their session and returning them to the homepage \\ 

    \textbf{Input} & User clicks logout button \\ 

    \textbf{Exp. Output} & The system invalidates the session and redirects the user to the homepage \\ 

    \textbf{Act. Output} & The system invalidates the session and redirects the user to the homepage  \\ 
    
    \small \textbf{Result} & Pass\\ 
  }
\end{center}

\subsection{FR214}
The system shall allow users to create a home group using the home name, address, and number of roommates.
\renewcommand{\arraystretch}{1.5}
\begin{center}
  \testcase{
    \textbf{FST-UAHM-4} & \textbf{Create Home Instance}\\\hline

    \textbf{Description} & Tests if the system allows users to create a new home group with specified details \\ 

    \textbf{Input} & Home name, address, and number of roommates to their respective fields \\ 

    \textbf{Exp. Output} & A new home group instance is created with the specified details, and the user is added as a member \\ 

    \textbf{Act. Output} & A new home group instance is created with the specified details, and the user is added as a member  \\ 
    
    \small \textbf{Result} & Pass\\ 
  }
\end{center}

\subsection{FR215}
The system shall allow users to invite other users to join their home group.

\subsection{FR216}
The system shall allow users to view the list of users in their home group.

\subsection{FR217}
The system shall allow users to remove users from their home
group.

\subsection{FR218}
The system shall allow users to leave their home group.

\subsection{FR221}
The system shall allow users to configure the ChatBot settings to
include or exclude messages corresponding to chore schedule, cleanliness
manager, and bill splitter.

\subsection{FR222}
The ChatBot shall send reminders to the group chat about
upcoming chores and events in the schedule 2 days in advance.

\subsection{FR223}
The ChatBot shall send notifications to the group chat about new
shared living space cleanliness scores immediately after an event is added to
the cleanliness manager page.

\subsection{FR224}
The ChatBot shall send notifications to the group chat about new
shared expenses added to the bill splitter page immediately after its
addition.

\subsection{FR231}
The system shall evaluate the cleanliness of the shared living space before and after a user enters and exits the space.

\subsection{FR232}

The system shall display the current cleanliness score of the shared living space.
\subsection{FR233}

The system shall display the detected messes in the shared living space.
\subsection{FR234}

The system shall allow users to view the cleanliness score of other users.

\subsection{FR235}
The system shall allow users to view the history of cleanliness scores and detected messes.

\subsection{FR241}
The system shall allow users to add a new chore to the schedule.

\subsection{FR242}
The system shall allow users to add a new event to the schedule.

\subsection{FR243}
The system shall allow users to input chore and event details (name, description, time, frequency, assigned users, etc.).

\subsection{FR244}
The system shall allow users to edit and delete chores and events.

\subsection{FR245}
The system shall allow users to view the schedule and mark chores as complete.

\subsection{FR251}
The system shall allow users to add a new expense to the bill splitter and notify the involved users.

\subsection{FR252}
The system shall allow users to view what they owe other housemates.

\subsection{FR253}
The system shall allow users to view what they owe others.

\subsection{FR254}
The system shall allow users to mark expenses as paid.

\subsection{FR255}
The system shall calculate debts in order to minimize the amount of transactions required between housemates.


\section{Nonfunctional Requirements Evaluation}
\subsection{Usability}
\subsubsection{NFR237}
The system shall declare an instances of someone altering a room
finished one there has been no activity in the room for a designated period
of time.
\subsubsection{NFR242}
The calendar system shall display all calendar events to users in
their time zone.

\subsection{Performance}
\subsubsection{NFR214}
The system should be able to authenticate a user with a median
response time of under 1 second.
\subsubsection{NFR234} Photos captured with the system will be in a quality high
enough to differentiate objects within frame.
\subsubsection{NFR235}
The system shall process image data in under 30 minutes.

\subsection{Privacy and Security}
\subsubsection{NFR211}
All data related to authentication must be encrypted in both
transit and in storage.
\subsubsection{NFR212}
Error messages related to authentication should not disclose
sensitive details such as ”Incorrect Password”.
\subsubsection{NFR213}
All data related to houses such as addresses and residents should
be encrypted in transit and in rest.
\subsubsection{NFR221}
The chatbot shall not disclose any sensitive information in its
messages such as addresses or full names.
\subsubsection{NFR231}
The system shall not record users.
\subsubsection{NFR232}
The system shall not capture images of users.
\subsubsection{NFR233}
The system shall encrypt and securely store all images of homes.
\subsubsection{NFR244}
The calendar system shall encrypt all events stored.
\subsubsection{NFR251}
The Bill Splitter system shall encrypt all events stored.


\subsection{etc.(MAYBE RENAME TO "OTHER")}
\subsubsection{NFR222}
The chatbot shall not send users too frequently to prevent
annoying users.
\subsubsection{NFR236}
The system shall not report false events which accuse someone of
reducing the cleanliness score of an environment.
\subsubsection{NFR241}
The calendar system shall store all calendar events in UTC.
\subsubsection{NFR243}
The calendar system shall have a granularity of 5 minutes.
\subsubsection{NFR252}
The Bill Splitter shall allow users to record numerical values of
prices with a granularity of two decimal places.

\section{Comparison to Existing Implementation}	

This section will not be appropriate for every project.

\section{Unit Testing}
Unit testing for this project is defined as a way of testing the smallest piece of code that can be logically isolated. Logical isolation means to isolate a piece of code based on a particular function. In this project, unit testing was performed on the user-facing application of the project while the other codebases of the project will be tested with other methods.\\\\For context on the this section of the report, the client-facing application was built using the following tools:
\begin{itemize}
\item \textbf{Next.js} as the JavaScript framework for both front-end and back-end
\item \textbf{ReactQuery} for data synchronization and caching
\end{itemize}
and is broken up into different pages based on application features such as:
\begin{itemize}
\item Cleanliness management system
\item Chore scheduling system
\item Bill splitting system
\item ...
\end{itemize}
While it is recommended to test as much of your code as possible. Due to time constraints and generally better use of our time, the team developed unit tests only for the critical components (Referring to React components) of our application, while other components that exist for purposes such as reusable styling or wrappers were ignoring. In general, unit tests were written with the objective of:
\begin{itemize}
\item Ensure key UI components were rendering
\item Ensuring the UI displayed information fetched from the back-end
\item Ensuring inputs and buttons on the UI are able to be interacted with
\end{itemize}
In the front-end, 105 different tests were written for over 20+ components resulting in the following code-coverage report.


% Define new column types with smaller font
\newcolumntype{S}[1]{>{\small\raggedright\arraybackslash}p{#1}}
\newcolumntype{U}[1]{>{\small\raggedright\arraybackslash}p{#1}}

\newgeometry{left=5mm,right=5mm}
\begin{longtable}{|S{0.3\linewidth} | >{\centering\arraybackslash}p{0.1\linewidth} | >{\centering\arraybackslash}p{0.1\linewidth} | >{\centering\arraybackslash}p{0.1\linewidth} | >{\centering\arraybackslash}p{0.1\linewidth} | U{0.1\linewidth}|}
    \caption{\bf Front-end Code Coverage Report} \label{tab:coverage} \\
    \hline
    \textbf{File} & \textbf{\% Stmts} & \textbf{\% Branch} & \textbf{\% Funcs} & \textbf{\% Lines} & \textbf{Uncovered Line \#s} \\
    \hline
    \endfirsthead
    
    \hline
    \textbf{File} & \textbf{\% Stmts} & \textbf{\% Branch} & \textbf{\% Funcs} & \textbf{\% Lines} & \textbf{Uncovered Line \#s} \\
    \hline
    \endhead
    
    \hline
    \endfoot
    
    \hline
    \endlastfoot

    All files & 82.9  & 93.24  & 58.87  & 82.9  & \\ \hline
    app & 100 & 100 & 100 & 100 & \\ \hline
    \quad page.tsx & 100 & 100 & 100 & 100 & \\ \hline
    app/(main)/bill-splitter/components & 83.37 & 100 & 60 & 83.37 & \\ \hline
    \quad debtsTable.tsx & 83.72 & 100 & 25 & 83.72 & 26--32, 34--38, 76--77 \\ \hline
    \quad historyTable.tsx & 100 & 100 & 100 & 100 & \\ \hline
    \quad loansTable.tsx & 63.82 & 100 & 57.14 & 63.82 & 22--67, 91--93, 95--96 \\ \hline
    \quad summaryCard.tsx & 100 & 100 & 100 & 100 & \\ \hline
    \quad summaryCardStub.tsx & 100 & 100 & 100 & 100 & \\ \hline
    app/(main)/bill-splitter/hooks & 33.33 & 100 & 12.5 & 33.33 & \\ \hline
    \quad patchOwe.ts & 55.26 & 100 & 50 & 55.26 & 5--21 \\ \hline
    \quad useBillHistory.ts & 19.44 & 100 & 0 & 19.44 & 6--29, 32--36 \\ \hline
    \quad useBills.ts & 26.08 & 100 & 0 & 26.08 & 5--16, 19--23 \\ \hline
    \quad useOwes.ts & 26.08 & 100 & 0 & 26.08 & 5--16, 19--23 \\ \hline
    app/(main)/chatbot & 95.62 & 80 & 50 & 95.62 & \\ \hline
    \quad page.tsx & 95.62 & 80 & 50 & 95.62 & 81--85, 101 \\ \hline
    app/(main)/chatbot/components & 95.74 & 33.33 & 50 & 95.74 & \\ \hline
    \quad chatbot-setting-stub.tsx & 95.74 & 33.33 & 50 & 95.74 & 21--22 \\ \hline
    app/(main)/chatbot/hooks & 24.39 & 100 & 0 & 24.39 & \\ \hline
    \quad useActivateChatbot.ts & 24.39 & 100 & 0 & 24.39 & 9--24, 27--41 \\ \hline
    app/(main)/dashboard/components & 100 & 100 & 100 & 100 & \\ \hline
    \quad dashboard-cards.tsx & 100 & 100 & 100 & 100 & \\ \hline
    app/(main)/house-settings/components & 100 & 100 & 100 & 100 & \\ \hline
    \quad create-note-modal.tsx & 100 & 100 & 100 & 100 & \\ \hline
    \quad edit-house-modal.tsx & 100 & 100 & 100 & 100 & \\ \hline
    \quad house-invites.tsx & 100 & 100 & 100 & 100 & \\ \hline
    \quad house-notes.tsx & 100 & 100 & 100 & 100 & \\ \hline
    \quad invite-user-modal.tsx & 100 & 100 & 100 & 100 & \\ \hline
    app/(main)/house-settings/hooks & 35 & 100 & 20 & 35 & \\ \hline
    \quad useRemoveRoomate.ts & 35 & 100 & 20 & 35 & 5--15, 22--23, 25--32, 34--38 \\ \hline
    app/(main)/profile & 87.22 & 80 & 60 & 87.22 & \\ \hline
    \quad page.tsx & 87.22 & 80 & 60 & 87.22 & 25--31, 39--43, 94--101, 159--161 \\ \hline
    app/(main)/schedule & 23.07 & 100 & 0 & 23.07 & \\ \hline
    \quad adapters.ts & 23.07 & 100 & 0 & 23.07 & 4--13 \\ \hline
    app/(main)/schedule/components & 98.76 & 95.74 & 100 & 98.76 & \\ \hline
    \quad chore-history.tsx & 100 & 100 & 100 & 100 & \\ \hline
    \quad create-chore-modal.tsx & 99.41 & 85.71 & 100 & 99.41 & 143 \\ \hline
    \quad pending-chores.tsx & 100 & 100 & 100 & 100 & \\ \hline
    \quad pending-item.tsx & 100 & 100 & 100 & 100 & \\ \hline
    \quad schedule-item.tsx & 95.58 & 92 & 100 & 95.58 & 64--69, 76--77 \\ \hline
    \quad schedule.tsx & 100 & 100 & 100 & 100 & \\ \hline
    app/(main)/schedule/hooks & 33.48 & 100 & 0 & 33.48 & \\ \hline
    \quad useCreateChore.ts & 14.28 & 100 & 0 & 14.28 & 5--17, 20--42 \\ \hline
    \quad useDeleteChore.ts & 14.63 & 100 & 0 & 14.63 & 5--16, 19--41 \\ \hline
    \quad useGetAllActivities.ts & 36.36 & 100 & 0 & 36.36 & 7--15, 18--22 \\ \hline
    \quad useGetAllCompletedChores.ts & 69.76 & 100 & 0 & 69.76 & 29--36, 39--43 \\ \hline
    \quad useGetCompletedChores.ts & 50 & 100 & 0 & 50 & 12--19, 22--26 \\ \hline
    \quad useUpdateCompletedChore.ts & 24 & 100 & 0 & 24 & 11--23, 26--50 \\ \hline
    app/auth/hooks & 47.36 & 100 & 0 & 47.36 & \\ \hline
    \quad useUser.tsx & 47.36 & 100 & 0 & 47.36 & 19--38 \\ \hline
    components & 100 & 100 & 85.71 & 100 & \\ \hline
    \quad loading.tsx & 100 & 100 & 100 & 100 & \\ \hline
    \quad modal.tsx & 100 & 100 & 100 & 100 & \\ \hline
    \quad mutate-loading.tsx & 100 & 100 & 100 & 100 & \\ \hline
    \quad query-provider.tsx & 100 & 100 & 100 & 100 & \\ \hline
    \quad roommates-table.tsx & 100 & 100 & 50 & 100 & \\ \hline
    components/ui & 97.69 & 88.23 & 100 & 97.69 & \\ \hline
    \quad button.tsx & 100 & 50 & 100 & 100 & 42 \\ \hline
    \quad card.tsx & 100 & 100 & 100 & 100 & \\ \hline
    \quad checkbox.tsx & 100 & 100 & 100 & 100 & \\ \hline
    \quad dialog.tsx & 100 & 100 & 100 & 100 & \\ \hline
    \quad form.tsx & 92.26 & 92.85 & 100 & 92.26 & 50--51, 123--133 \\ \hline
    \quad input.tsx & 100 & 100 & 100 & 100 & \\ \hline
    \quad label.tsx & 100 & 100 & 100 & 100 & \\ \hline
    \quad table.tsx & 100 & 100 & 100 & 100 & \\ \hline
    hooks & 45.16 & 75 & 33.33 & 45.16 & \\ \hline
    \quad useGetHouse.ts & 24.13 & 100 & 0 & 24.13 & 6--22, 25--29 \\ \hline
    \quad useRoommates.ts & 63.33 & 60 & 100 & 63.33 & 9--11, 13--20 \\ \hline
    \quad useToast.ts & 45.5 & 100 & 14.28 & 45.5 & 27--30, 59--72, 75--125, 131--136, 140--167 \\ \hline
    lib & 61.53 & 100 & 50 & 61.53 & \\ \hline
    \quad utils.ts & 61.53 & 100 & 50 & 61.53 & 9--13 \\ \hline
    lib/constants & 100 & 100 & 100 & 100 & \\ \hline
    \quad index.ts & 100 & 100 & 100 & 100 & \\ \hline
    lib/supabase & 44.44 & 100 & 0 & 44.44 & \\ \hline
    \quad browser.ts & 44.44 & 100 & 0 & 44.44 & 5--9 \\ \hline

\end{longtable}
\restoregeometry


\section{Changes Due to Testing}

\wss{This section should highlight how feedback from the users and from 
the supervisor (when one exists) shaped the final product.  In particular 
the feedback from the Rev 0 demo to the supervisor (or to potential users) 
should be highlighted.}\\
\textbf{FR211} - Change description to use Google account instead of using name, email and password.\\
\textbf{FST-UAHM-1} - See above description. \\
\textbf{FST-UAHM-2} - Change input field of testcase to using Google account instead of email and password. \\

\subsection{Changes Due to Unit Testing}
Unit testing discovered several bugs and anti-patterns in the front-end codebase that were resolved. Some examples of the discovered flaws include:
\begin{itemize}
\item Invalid references for form inputs and form labels. Leading to input labels not focusing their corresponding form input.
\item Unneccessary props on React components.
\item Missing semantic HTML attributes such as "role" on various items. Making the app less accessible and more difficult for the testing library to find components.
\item Checking for lists to be empty with <LIST VARIABLE NAME>. This resulted in empty lists still being evaluated to "True" because in JS empty lists are truthy and resulted in code logic that was supposed to trigger when the list was empty to not work as expected.
\item Displaying times without the timezone. This was detected because the testing suite failed on various environments due to the timezone printing differently on machines with different timezones.
\end{itemize}
The code coverage reports shows some trends, mainly that all most if not all of our hooks have poor code coverage rates. This is expected and acceptable because the purpose of the hooks is to fetch data from the back-end and display it on the front-end and since we were mocking the back-end data, most of the hooks' functionality were not used.

\section{Automated Testing}
Automated tests include the unit tests and the cleanliness detection system's test samples. The front-end tests were additionally automated to run on push and on merge requests to main and dev. These automated tests on GitHub in conjunction with branch rules that prevent merging branches which fail tests ensure that no changes the fail tests will make it to the production environment.
		
\section{Trace to Requirements}
		
\section{Trace to Modules}		

\section{Code Coverage Metrics}

\bibliographystyle{plainnat}
\bibliography{../../refs/References}

\newpage{}
\section*{Appendix --- Reflection}

The information in this section will be used to evaluate the team members on the
graduate attribute of Reflection.

\input{../Reflection.tex}

\begin{enumerate}
  \item What went well while writing this deliverable? 
  \item What pain points did you experience during this deliverable, and how
    did you resolve them?
  \item Which parts of this document stemmed from speaking to your client(s) or
  a proxy (e.g. your peers)? Which ones were not, and why?
  \item In what ways was the Verification and Validation (VnV) Plan different
  from the activities that were actually conducted for VnV?  If there were
  differences, what changes required the modification in the plan?  Why did
  these changes occur?  Would you be able to anticipate these changes in future
  projects?  If there weren't any differences, how was your team able to clearly
  predict a feasible amount of effort and the right tasks needed to build the
  evidence that demonstrates the required quality?  (It is expected that most
  teams will have had to deviate from their original VnV Plan.)
\end{enumerate}

\end{document}
