\documentclass[12pt, titlepage]{article}

\usepackage{booktabs}
\usepackage{tabularx}
\usepackage{hyperref}
\hypersetup{
    colorlinks,
    citecolor=black,
    filecolor=black,
    linkcolor=red,
    urlcolor=blue
}
\usepackage[round]{natbib}
\usepackage{longtable}
\usepackage[shortlabels]{enumitem}
\usepackage{hyperref}
\hypersetup{
    colorlinks,
    citecolor=blue,
    filecolor=black,
    linkcolor=red,
    urlcolor=blue
}
\usepackage{amsmath}
\usepackage{array}
\usepackage{graphicx}

\input{../Comments}
%% Common Parts

\newcommand{\progname}{Room8} % PUT YOUR PROGRAM NAME HERE
\newcommand{\authname}{Team 19
\\ Mohammed Abed
\\ Maged Armanios
\\ Jinal Kasturiarachchi
\\ Jane Klavir
\\ Harshil Patel} % AUTHOR NAMES

\usepackage{hyperref}
    \hypersetup{colorlinks=true, linkcolor=blue, citecolor=blue, filecolor=blue,
                urlcolor=blue, unicode=false}
    \urlstyle{same}
                                


\begin{document}

\title{Verification and Validation Report: Room8} 
\author{\authname}
\date{\today}
	
\maketitle

\pagenumbering{roman}

\section{Revision History}

\begin{tabularx}{\textwidth}{p{3cm}p{2cm}X}
\toprule {\bf Date} & {\bf Version} & {\bf Notes}\\
\midrule
Date 1 & 1.0 & Notes\\
Date 2 & 1.1 & Notes\\
\bottomrule
\end{tabularx}

~\newpage

\section{Symbols, Abbreviations and Acronyms}

\renewcommand{\arraystretch}{1.2}
\begin{tabular}{l l} 
  \toprule		
  \textbf{Acronym} & \textbf{description}\\
  \midrule 
  SRS & Software Requirements Specification\\
  VnV & Verification and Validation\\
  CI/CD & Continuous Integration and Continuous deployment\\   
  API & Application Programming Interface\\
  Exp. & Expected\\
  Act. & Actual\\
  \bottomrule
\end{tabular}\\

\wss{symbols, abbreviations or acronyms -- you can reference the SRS tables if needed}

\newpage

\tableofcontents

\listoftables %if appropriate

\listoffigures %if appropriate

\newpage

\pagenumbering{arabic}

This document ...

\section{Functional Requirements Evaluation}

\subsection{FR211}
The system shall allow users to create an account using their Google account.
\renewcommand{\arraystretch}{1.5}
\begin{center}
  \testcase{
    \textbf{FST-UAHM-1} & \textbf{Profile Creation}\\\hline

    \textbf{Description} & Tests if the system is able to handle creating user 			profiles in the system from frontend inputs \\ 

    \small \textbf{Input} & Valid Google account sign in \\ 

    \small \textbf{Exp. Output} & Room8 account created and Frontend redirects user to the dashboard \\ 

    \small \textbf{Act. Output} & Room8 account created and Frontend redirects user to the dashboard\\ 
    
    \small \textbf{Result} & Pass\\ 

  }
\end{center}
\subsection{FR212}
The system shall allows users to log in to their account.
\renewcommand{\arraystretch}{1.5}
\begin{center}
  \testcase{
    \textbf{FST-UAHM-2} & \textbf{System Login}\\\hline

    \textbf{Description} & Tests if the system is able to recognize a user profile and to authenticate them to the system \\ 

    \textbf{Input} & Valid Google account matching the profile in the system \\ 

    \textbf{Exp. Output} & User profile is authenticated with a new session entry. Frontend redirects user to the dashboard \\ 

    \textbf{Act. Output} & User profile is authenticated with a new session entry. Frontend redirects user to the dashboard  \\ 
    
    \small \textbf{Result} & Pass\\ 
  }
\end{center}

\subsection{FR213}
The system shall allows users to log out of their account.
\renewcommand{\arraystretch}{1.5}
\begin{center}
  \testcase{
    \textbf{FST-UAHM-3} & \textbf{System Logout}\\\hline

    \textbf{Description} & Tests if the system logs the user out, invalidating their session and returning them to the homepage \\ 

    \textbf{Input} & User clicks logout button \\ 

    \textbf{Exp. Output} & The system invalidates the session and redirects the user to the homepage \\ 

    \textbf{Act. Output} & The system invalidates the session and redirects the user to the homepage  \\ 
    
    \small \textbf{Result} & Pass\\ 
  }
\end{center}

\subsection{FR214}
The system shall allow users to create a home group using the home name, address, and number of roommates.
\renewcommand{\arraystretch}{1.5}
\begin{center}
  \testcase{
    \textbf{FST-UAHM-4} & \textbf{Create Home Instance}\\\hline

    \textbf{Description} & Tests if the system allows users to create a new home group with specified details \\ 

    \textbf{Input} & Home name, address, and number of roommates to their respective fields \\ 

    \textbf{Exp. Output} & A new home group instance is created with the specified details, and the user is added as a member \\ 

    \textbf{Act. Output} & A new home group instance is created with the specified details, and the user is added as a member  \\ 
    
    \small \textbf{Result} & Pass\\ 
  }
\end{center}

\subsection{FR215}
The system shall allow users to invite other users to join their home group.

\subsection{FR216}
The system shall allow users to view the list of users in their home group.

\subsection{FR217}
The system shall allow users to remove users from their home
group.

\subsection{FR218}
The system shall allow users to leave their home group.

\subsection{FR221}
The system shall allow users to configure the ChatBot settings to
include or exclude messages corresponding to chore schedule, cleanliness
manager, and bill splitter.

\subsection{FR222}
The ChatBot shall send reminders to the group chat about
upcoming chores and events in the schedule 2 days in advance.

\subsection{FR223}
The ChatBot shall send notifications to the group chat about new
shared living space cleanliness scores immediately after an event is added to
the cleanliness manager page.

\subsection{FR224}
The ChatBot shall send notifications to the group chat about new
shared expenses added to the bill splitter page immediately after its
addition.

\subsection{FR231}
The system shall evaluate the cleanliness of the shared living space before and after a user enters and exits the space.

\subsection{FR232}

The system shall display the current cleanliness score of the shared living space.
\subsection{FR233}

The system shall display the detected messes in the shared living space.
\subsection{FR234}

The system shall allow users to view the cleanliness score of other users.

\subsection{FR235}
The system shall allow users to view the history of cleanliness scores and detected messes.

\subsection{FR241}
The system shall allow users to add a new chore to the schedule.

\subsection{FR242}
The system shall allow users to add a new event to the schedule.

\subsection{FR243}
The system shall allow users to input chore and event details (name, description, time, frequency, assigned users, etc.).

\subsection{FR244}
The system shall allow users to edit and delete chores and events.

\subsection{FR245}
The system shall allow users to view the schedule and mark chores as complete.

\subsection{FR251}
The system shall allow users to add a new expense to the bill splitter and notify the involved users.

\subsection{FR252}
The system shall allow users to view what they owe other housemates.

\subsection{FR253}
The system shall allow users to view what they owe others.

\subsection{FR254}
The system shall allow users to mark expenses as paid.

\subsection{FR255}
The system shall calculate debts in order to minimize the amount of transactions required between housemates.


\section{Nonfunctional Requirements Evaluation}
\subsection{Usability}
\subsubsection{NFR237}
The system shall declare an instances of someone altering a room
finished one there has been no activity in the room for a designated period
of time.
\subsubsection{NFR242}
The calendar system shall display all calendar events to users in
their time zone.

\subsection{Performance}
\subsubsection{NFR214}
The system should be able to authenticate a user with a median
response time of under 1 second.
\subsubsection{NFR234} Photos captured with the system will be in a quality high
enough to differentiate objects within frame.
\subsubsection{NFR235}
The system shall process image data in under 30 minutes.

\subsection{Privacy and Security}
\subsubsection{NFR211}
All data related to authentication must be encrypted in both
transit and in storage.
\subsubsection{NFR212}
Error messages related to authentication should not disclose
sensitive details such as ”Incorrect Password”.
\subsubsection{NFR213}
All data related to houses such as addresses and residents should
be encrypted in transit and in rest.
\subsubsection{NFR221}
The chatbot shall not disclose any sensitive information in its
messages such as addresses or full names.
\subsubsection{NFR231}
The system shall not record users.
\subsubsection{NFR232}
The system shall not capture images of users.
\subsubsection{NFR233}
The system shall encrypt and securely store all images of homes.
\subsubsection{NFR244}
The calendar system shall encrypt all events stored.
\subsubsection{NFR251}
The Bill Splitter system shall encrypt all events stored.


\subsection{etc.(MAYBE RENAME TO "OTHER")}
\subsubsection{NFR222}
The chatbot shall not send users too frequently to prevent
annoying users.
\subsubsection{NFR236}
The system shall not report false events which accuse someone of
reducing the cleanliness score of an environment.
\subsubsection{NFR241}
The calendar system shall store all calendar events in UTC.
\subsubsection{NFR243}
The calendar system shall have a granularity of 5 minutes.
\subsubsection{NFR252}
The Bill Splitter shall allow users to record numerical values of
prices with a granularity of two decimal places.

\section{Comparison to Existing Implementation}	

This section will not be appropriate for every project.

\section{Unit Testing}

\section{Changes Due to Testing}

\wss{This section should highlight how feedback from the users and from 
the supervisor (when one exists) shaped the final product.  In particular 
the feedback from the Rev 0 demo to the supervisor (or to potential users) 
should be highlighted.}\\
\textbf{FR211} - Change description to use Google account instead of using name, email and password.\\
\textbf{FST-UAHM-1} - See above description. \\
\textbf{FST-UAHM-2} - Change input field of testcase to using Google account instead of email and password. \\

\section{Automated Testing}
		
\section{Trace to Requirements}
		
\section{Trace to Modules}		

\section{Code Coverage Metrics}

\bibliographystyle{plainnat}
\bibliography{../../refs/References}

\newpage{}
\section*{Appendix --- Reflection}

The information in this section will be used to evaluate the team members on the
graduate attribute of Reflection.

\input{../Reflection.tex}

\begin{enumerate}
  \item What went well while writing this deliverable? 
  \item What pain points did you experience during this deliverable, and how
    did you resolve them?
  \item Which parts of this document stemmed from speaking to your client(s) or
  a proxy (e.g. your peers)? Which ones were not, and why?
  \item In what ways was the Verification and Validation (VnV) Plan different
  from the activities that were actually conducted for VnV?  If there were
  differences, what changes required the modification in the plan?  Why did
  these changes occur?  Would you be able to anticipate these changes in future
  projects?  If there weren't any differences, how was your team able to clearly
  predict a feasible amount of effort and the right tasks needed to build the
  evidence that demonstrates the required quality?  (It is expected that most
  teams will have had to deviate from their original VnV Plan.)
\end{enumerate}

\end{document}